\documentclass[12pt]{article}
\usepackage[utf8]{inputenc}

\title{FastFlow Note}
\author{Adam}
\date{\today}
\begin{document}

\maketitle

\section{General Tech Tips}
\begin{enumerate}
    \item \large{Async Functions}
    \\- You can only use \textbf{await} inside functions created with \textbf{async def}.

    \item \large{Powerful SQLAlchemy}
    \\- SQLAlchemy is a powerful SQL toolkit and Object-Relational Mapping (ORM) library for Python.
    \\- Use Object-Relational Mapping (ORM) in Python to access SQL databases.
    \\- Still available for raw SQL query: 
    \\e.g., \textbf{result = db.execute("SELECT * FROM users WHERE id = :id", {"id": 1})}

    \item \large{Success and Message Endpoints in FastAPI to Next.js (use JSONResponse!!!)}
    \\- Code:
    \begin{verbatim}
    from fastapi.responses import JSONResponse
    from fastapi import status, HTTPException

    return JSONResponse(
        status_code=status.HTTP_404_NOT_FOUND,
        content={"success": False, "message": f"Workout with id {workout_id} not found or does not belong to you."}
    )
    \end{verbatim}

    \item \large{The Use of \_\_init\_\_.py}
    \\- An \textbf{\_\_init\_\_.py} file (even if empty) marks a directory as a Python package.
    \\- It tells Python that the directory should be treated as a module package.
    \\- Allows you to import modules from that directory.
    \\- Used for package initialization and can contain code that runs when the package is imported.

    \item \large{text}
    \\- text

\end{enumerate}

\section{Specific Stack}
Specific details here.

\end{document}
